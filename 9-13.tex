\chapter{Homogeneous dynamics}

\section{Introduction: Homogeneous dynamics}

Homogeneous dynamics is about a special kind of dynamical system. We start with $G$, a locally compact topological group. Think of 2 examples: $G=\R^n$ and $G=\SL_n(\R)=\set{g\in\cal M_{n\times n}(\R)}{\det g=1}$. 

Consider $\Ga<G$ discrete with finite covolume. Consider the quotient manifold $\Ga \bs G \cil G$, which has a natural action by $G$,
\begin{align}
h.\Ga g &=\Ga gh^{-1}.
\end{align}
It has finite covolume if $\Ga \bs G$ carries a finite $G$-invariant Borel measure. It will always be explicit for us; we can write it down in coordinates.
It always carries a left-invariant and right-invariant measure called the Haar measure, unique up to constant. The quotient group also carries a invariant measure but it doesn't have to be finite. 

The Haar measure on $\R^n$ is the Lebesgue measure.

\begin{exr} Write a Haar measure on $\GL_n(\R)$ in coordinates; write in terms of $g=(x_{ij})_{ij}$. 
\end{exr}
\begin{rem}
If $\Ga\bs G$ is compact, then $\Ga$ is a lattice in $G$ ($\Ga < G$). 
\end{rem}

\begin{ex}
Let $\Ga = \Z^n<\R^n=G$. 
Then $\Ga \bs G = \Z^n\bs \R^n\cong \T^n$.
%lattice: quotient compact
Note $\T^1\cong \bS^1$. 
The quotient can be represented by a fundamental domain, $[0,1]$, where 0 and 1 are identified.
\end{ex}
\begin{ex}
Another example: $\Ga = \SL_n(\Z) = \set{g\in \cal M_{n\times n} (\Z)}{\det g=1}$ is a lattice in $\SL_n(\R)$.  
The quotient is not compact but has finite volume; it's a weird object. It has something that goes to infinity but gets thinner and thinner.
It's a moduli problem of some sort.
%never compact, weird object
%not compact but has finite volume. 
%something that goes to infinity but gets thinner and thinner.
\end{ex}

Dynamics: Let $H<G$ be a closed subgroup. Then $H\cir \Ga \bs G$. 
We want to understand all orbits of this group, or as many as we can.

Consider a line in $\R^2$. If we quotient by $\Z^2$, the line folds into itself. This folding can be complicated. The closure of each orbit is either either periodic (a subtorus, if the slope is rational) or everything. This is a simple result we try to imitate in a more complicated setting.

Our aim is to understand orbit closures of $H \cir \Ga \bs G$ and $H$-invariant probability measures on $\Ga\bs G$. 

The classic motivation was from geometry: flows on negatively curved manifolds. In the 80's, due to Margulis, the subject became related to number theory, and exploded exponentially.

Margulis solved the following problem, for which the field became famous.
\begin{conj}[Oppenheim]
Let $Q(x_1,\ldots, x_n) = x^\top Gx = \sum g_{ij}x_ix_j$ be an irrational indefinite non-degenerate quadratic form in $n\ge 3$ variables.
Then $Q(\Z^n_{\text{prim}})$ is dense in $\R$. ($\Z^n_{\text{prim}}$ is the set of points with no common divisors.)
\end{conj}
Davenport proved this for $n\ge 9$. It was hard to push it down to 3. 

The solution depends on this observation due to Raghunatan and Margulis. The conjecture is equivalent to the following statement:
Consider $\SO(Q) = \set{g\in \SL_n(\Z
%\R
)}{Q(g.x) = Q(x)}$. Note $\SO(Q)\cir \SL_n(\Z) \bs \SL_n(\R)$. 
(To see this, note $Q(\Z^n)$ is invariant under $\SL_n(\Z)$.)
\begin{conj}
Every orbit of $\SO(Q)$ on $\SL_n(\Z)\bs \SL_n(\R)$ is either unbounded or it is compact and carries a finite $\SO(Q)$-invariant measure. 
\end{conj}
You cannot have bounded orbits which are not compact. This dichotomy between 2 types of orbits is very close to the original conjecture.

This type of proof gives non-effective results at first. It's a soft statement. Since then it has been made effective to some extent.

%examples of personal taste
%not not important, just less close to my heart

The following example can be solved using homogeneous dynamics, or 
harmonic analysis and automorphic forms. The methods are related.
\begin{ex}
Let $V_d$ be the variety of matrices with $\det X=d$, $d\in \N$, so
\begin{align}
V_d(\R) &=\set{X\in \cal M_{n\times n}(\R)}{\det X=d}.
\end{align}
Question: Fix a ``nice'' bounded open set $\Om\subeq V_1(\R)$. How many integral points are there in $\Om$---how does this grow as $d\to \iy$?
\begin{align}
\text{as }d\to \iy, \quad |d^{-1/n}V(\Z) \cap \Om|&\sim ?
\end{align}
The answer is that 
\begin{align}
\fc{|d^{-1/n}V(\Z)\cap \Om_1|}{|d^{-1/n}V(\Z)\cap \Om_2|} \to \fc{m(\Om_1)}{m(\Om_2)} 
\end{align}
for the Haar measure $m$. %an explicit measure $m$.
It's easy to calculate the rate of convergence.
%If you modify the integral model what happens to the solutions?

Observe that $V_d$ is homogeneous, $G=\SL_n(\R)\times \SL_n(\R)\cir V_d(\R)$ by 
\begin{align}
(g_1,g_2).X &= g_1Xg_2^{-1}.
\end{align}
The action is transitive and $G(\Z)$ stabilizes $V_d(\Z)$. There's only finitely many $G(\Z)$ orbits on $V_d(\Z)$ (due to Borel, Harish-Chandra). 
To see this, note each matrix has a Smith normal form: for all $M\in \cal M_{n\times n}(\Z)$ there exists $S,T\in \GL_n(\Z)$ and $d_1|\cdots |d_n$ such that 
$M=S\mattn{d_1}{}{}{}{\ddots}{}{}{}{d_n}T$. There are finitely many orbits and you can classify them.

All points are conjugate, so to understand the stabilizer, fix your favorite point and calculate its stabilizer.
\begin{align}
\Stab_G (d^{-1/n} I_d) = \SL_n(\R)^\De \hra \SL_n(\R)\times \SL_n(\R). 
\end{align}
Understanding $Q$ is the same as studying ``periodic'' orbits of $\SL_n(\R)^\De\cir \SL_n(\Z)\bs \SL_n(\R) \times \SL_n(\Z) \bs \SL_n(\R)$. 
If a group acts transitively, the space looks like the group quotiented by the stabilizer.
\begin{align}
V_d(\R) &\cong \SL_n(\R)\times \SL_n(\R) /\SL_n(\R)^\De.
\end{align}•
%higher and higher volume become equidist wrt Haar measure
You can show this with automorphic forms, using a spectral gap, so don't be impressed by this example.
\end{ex}

The following is more impressive, and brings the world of $p$-adics into the game.
%very thin set in set of integral points.
\begin{ex}
Let $V(\R) = \bS^{n-1}(\R)\times \bS^{n-1}(\R) = \set{v,w\in \R^n}{\an{v,v}=1,\an{w,w}=1} $, $n\ge 3$. 
For $d\in \N, e\in \Z$. 
%for which values of d
\begin{align}
V_{d,e}(\Z) = \set{v,w\in \Z_{\text{prim}}^n}{\an{v,v}=d,\an{w,w}=d,\an{v,w}=e}.
\end{align}
A classical number theorist can check for which $d,e$ there are solutions. 
For $\Om\subeq V(\R)$ nice. Then the conjecture is that
%usual rot-inv measure
\begin{align}
\fc{\ab{d^{-1/2} V_{d,e}(\Z)\cap \Om}}{|V_{d,e}(\Z)|} \to m\times m(\Om)
\end{align}
where $m$ is the rotation invariant measure on the sphere.
%stack in positive codim
if $d\to \iy$, $|e|\to \iy$. 

This is known for $n\ge 4$ (certainly for $n\ge 6)$. For $n=3$ this is open, related to Fourier coefficients of modular forms, and the Andre-Orr conjecture.
\end{ex}
%shouldn't have finite set of accum points
For $n=3$ you can count up to Ziegler's bound; the  number of points on the sphere is a class number.
%For $e$ prime. Pairs where you can move from one to other from fixed element of the class group.
%extra action

$SO_n(\R) \times \SO_n(\R)\cir V(\R)$.  The stabilizer is conjugate  over $\R$ to $\SO_{n-1}(\R)^\De\hra \SO_n(\R) \times \SO_n(\R)$ (You need to rotate the same).
%sitting diagonally inside
%must rotate by the same rotation
$\SO_{n-1}(\R)^\De$ is compact, which is a big problem; dynamics are simple and there's nothing you can use. Why can you solve it using dynamics? Enlarge your original problems. Because equations are integral, consider not just integral but $p$-adic places. Stabilizers of points may be non-compact in the $p$-adic place. Use dynamics of the $p$-adic places.

Except for Oppenheim's conjecture, in all applications you use $p$-adic places in some deep way.

You can also use dynamics to count points. 

The next example is surprising. 
You generate a nontrivial integral point using dynamics. This is an example of a different nature.
% Just counting how many points there are can use 

\begin{ex}[Ellenberg-Venkatesh]
Let $(Z^n,Q),(Z^m,Q')$ be quadratic spaces (coordinate free version of quadratic form).

We try to represent one by another using more variables. In the simplest case, can we represent an integer by a given quadratic form; that's the $m=1$ case.
%coordinate free version of quadratic form

We say that $Q'$ is representable by $Q$ if there is a homomorphism of abelian groups $\ph:\Z^m\hra \Z^n$ such that $Q'=Q\circ\ph$. This is a fancy way to say the following: if you write in coordinates,
\begin{align}%y_1,\ldots, y_m
Q'(y) &= Q(A.y), \quad A\in M_{n\times m}(\Z).
\end{align}
\end{ex}

There are some obviously necessary conditions, the local conditions.
%It's a question in analysis
\begin{df}
$Q'$ is everywhere locally representable by $Q$ if $Q'\ot \R$ is representable by $Q\ot \R$ and for all $p$, and for all $p$, $Q'\ot \Z_p$ is representable by $Q\ot \Z_p$, or equivalently, for all $p^d$, $Q'\bmod {p^d}$ is representable over $\Z/p^d\Z$ by $Q\bmod{p^d}$. 
\end{df}
Is this sufficient, i.e., does the local-to-global principle hold? Not necessarily. 


\begin{thm}[Ellenberg-Venkatesh]
Let $Q$ be a positive definite form over $\Z^n$. There exists $C(Q)$ such that if $m\le n-7$ and $(\Z^m,Q')$ has square-free discriminant, $Q$ represents $Q'$ locally everywhere, and $\min_{\bS^{m-1}(\R)} Q' %(\R)
>C(Q)$, then $Q$ represents $Q'/\Z$. 
%big enough condition
\end{thm}
%integral points, set of rep, write one in terms of another.
The proof idea: it's known it has a rational point, you want to upgrade it to a integral point. Integrality can be checked locally, and so can be checked using dynamics.
%min over nonzero opints.
\begin{conj}
This holds if $n-m\ge 3$. 
\end{conj}
What happens if $n-m=2$ is interesting. It's unclear what to conjecture. The stabilizer becomes abelian; its nature changes. %quotient nonabelian.

There is another pair of examples, due to Venkatesh.
\begin{enumerate}
\item
A modular form on nonarithmetic group cannot have multiplicative coefficients. This is a nice application of homogeneous dynamics. See Kowalski's blog.
\item Elliptic curves: higher-order Heegner points, traces are nontrivial.
\end{enumerate}
Let $G$ be a locally compact second countable group, $G=\SL_n(\R)$. $G$ carries a left invariant Borel measure $m_L$ and a right invariant Borel measure $m_R$. I.e., for all $A$, for all $g\in G$, 
\begin{align}
m_L(gA)&=m_L(A).
\end{align}
By Riesz representation, this is equivalent to: for all $f\in C_c(G)$, 
\begin{align}
\int f(gx) \, dm_L(x) &= \int f(x) \, dm_L(x).
\end{align}
The measures are not always the same. $G$ is \vocab{unimodular} if $m_L=m_R$.
\begin{exr}
Show that $\SL_n(\R)$ is unimodular.
\end{exr}