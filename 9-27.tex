\blu{9-27}

%$\SL_n(\Z) \sub \SL_n(\R)$, a lattice, $\la_1:\SL_n(\Z)\bs \SL_n(\R)\cong \cal L_n^1 \to \R_{>0}$, $\la_1(\La) = \min_{0\ne v\in \La}\ve{v}_2$. 

We need one general fact before we dive deeper.

$\cal L_1^n$ is not compact. 

\begin{lem}[Mahler compactness criterion]
For $\si>0$ define $\cal L_n^1(\si)= \set{\La\in \cal L_n^1}{\la_1(\La) \ge \de}$. Then for all $\de>0$, $\cal L_n^1(\de)$ and for all $K\subeq \cal L_n^1$ compact there exists $\de_K>0$ such that $K\subeq \cal L_n^1(\de)$. 
\end{lem}
\begin{proof}
$\la_1$ is continuous, so restricted to a compact set $K$ it attains minimum $\de_K$, so $K\subeq \cal L_n^1(\de_K)$.

If $\La\in \cal L_n^1(\de)$, let $0\ne v_1\in \La$ be a vector of minimal length, and write $\La = \an{v_1,\ldots, v_n}_\Z$. %nothing happens to the first vector
This basis can be reduced using the LLL algorithm to a basis $v_1^*,\ldots, v_n^*$, $v_1=v_1^*$. 

We know that $a_1=\ve{v_1^*}\ge \de$, $a_2=\ve{v_2^*}\ge t \ve{v_1^*} = t\de$, and so on, $a_n=\ve{v_n^*}\ge t^{n-1}\de$.  %restricted to a compact set 
We also know $\prodo in \ve{v_i^*}=1$ because the lattice is unimodular.

Then 
\begin{align}
\rc{\prod_{i>1} \ve{v_i^*}}&\ge \ve{v_1^*}\ge \de\\
\rc{\prod_{i\ne 2} \ve{v_i^*}}&\ge 
\ve{v_2^*} &\ge td
\end{align}
so each one of the diagonal entries is inside a closed interval, so it is a compact set.
%These inequalities trace out a compact set.
%$v_1,\ldots$, LLL reduced, $v_1^*,\ldots$ Gram-Schmidt
%$\ga g = n a k$, $n\in N_s$, $a\in A_t$, k\in K$.
\end{proof}
%$A_t=\diag(a_1,\ldots, a_n)$, $\prod_i a_i=1$, for all $i$, \fc{a_{i+1}}{a_i}\ge t$.
%The noncompactness came from the fact there was no restriction on $a_i$. 
From sticking other groups in $\SL_n(\R)$ you can learn things about then. See e.g. the Borel-Harish-Chandra theorem. If you want to prove the class group is finite, use the geometry of numbers.

\subsection{$\SL_2(\R)$ and hyperbolic plane}

We discuss the relation of $\SL_2(\R)$ to hyperbolic geometry. 

$\SL_2(\R)$ is special.
There are some infinite sequence of algebraic groups which are simple. Special linear group, orthogonal groups, symplectic groups, and not much beyond that.
It is a special linear group, symplectic group, and group preserving a quadratic form. This collision of different properties makes it special.

Let $\bb H = \set{z\in \C}{\Re z>0}$. 

There is a nice action of $\GL_2(\C)$ on $\hat \C$,
\begin{align}
\matt abcd . z = \fc{az+b}{cz+d}.
\end{align}
It is 2-transitive. There is an action of $\SL_2(\R)\cir \bb H$.
Consider the tangent space $T\bb H = \bb H \times \C$. Give $\bb H$ a Riemann metric by giving each point of the tangent space an inner product: for $z\in \bb H$, $u,v\in \C\cong \{z\}\times \C$, define
\begin{align}
\an{u,v}_{z} & =\fc{\an{u,v}_{}}{(\Im z)^2}
\end{align}•
The curvature is constant $-1$. This is the unique hyperbolic plane.

We have actions $\SL_2(\R)\cir \bb H$ by isometries and $\SL_2(\R) \cir T\bb H$ by differentiation (?):
\begin{align}
\matt abcd . (z,u) &= \pa{\fc{az+b}{cz+d}, \fc{u}{(cz+d)^2}}.
\end{align}
The action of $\SL_2(\R)\cir \bb H$ is not faithful, $-I$ doesn't doesn't do anything.
\begin{df}
Let $R$ be a commutative ring. Define $\PSL_2(R) = Z(\SL_2(R))\bs \SL_2(R)$. 
\end{df}
Note that $Z(\SL_2(\R)) = \pm e = \mu_2$.
\begin{exr}
%This is not algebraic. 
No algebraic group represents $\PSL_2$. (Check what happens over an algebraically closed field and compare with $\PGL_2$.) 
\end{exr}
We know $\PSL_2\cir \bb H$ acts transitively. The stabilizer of $i$ is $\Stab_{\PSL_2(\R)} i = \mathrm{PSO}_2(\R)$. Thus
\begin{align}
\SL_2(\R)/\SO_2(\R) 
\xrc 
\PSL_2(\R)/\mathrm{PSO}_2(\R)
\xrc
\mathbb H.
\end{align}
%point and orientation
The unit tangent bundle is $T^1\bb H = \set{(z,u)}{\an{u,u}_z=1}$. Then $\SL_2(\R)\cir T^1\bb H$ simply transitively.

%see as top 
Topologically, $T^1\bb H$ is contractible to circle around a point, seen using Iwasawa decomposition.

%subset of Riemann sphere. 
The boundary of the upper half plane is line and point at infinity; that they seem disconnected is artificial.

The geodesics are vertical lines and semicircles perpendicular to $\R$. It's easy to show a vertical line is a geodesic. There is a unique geodesic between every two points; you can move the line to any other geodesic.

There is a geodesic flow $\R\cir T^n\bb H$. For all $(z,u)\in T^1\bb H$, there exists a unit speed geodesic $\ga:\R\to \bb H$, such that $\ga(0)=z$ and $d\ga|_0=u$. %$t_0(z,u) = (\ga(t),d\ga|_t)$. 

\begin{pr}
The geodesic flow coincides with the action of $A_+<\SL_2(\R)$, the group of diagonal matrices $A_T=\set{\matt{\la}{}{}{\la^{-1}}}{\la >0}$. Specifically, the time $t$ flow is $\matt{e^{\fc t2}}{}{}{e^{-\fc{t}2}}$. 
\end{pr}
Take a point on the real line; there are infinitely many geodesics coming out of it.
Going towards the point, the distance decays exponentially with the flow.
On the other side the distance grows exponentially. This is typical hyperbolic behavior, also responsible for interesting dynamics.

The $\SL_2(\R)$ invariant measure on $\bb H$ is $\fc{dx\,dy}{y^2}$, for $z=x+iy$. 
\subsection{Horospherical groups}

This is a completely general phenomenon. 
\begin{df}
Let $G$ be locally compact and $\si$-compact. 
Define $G_a^-$, the \vocab{stable horospherical subgroup},
\begin{align}
G>G_a^- & = \set{g\in G}{\limn a^n g a^{-n}=e}.
\end{align}
The \vocab{unstable horospherical subgroup} is
\begin{align}
G>G_a^+& = G_{a^{-1}}^-
\end{align}
Also define the parablic subgroups
\begin{align}
P_a^- &= \set{g\in G}{\limn a^n ga^{-n}\text{ exists}}\\
P_a^+ &= P_{a^{-1}}^-.
\end{align}•
%semisimple element
%parabolic subgroup
\end{df} 
If $G$ is abelian, this is just the whole group. If $G$ is compact, also nothing exciting can happen. The group is too simple. %rotations 
The noncommutativity of isometries over hyperbolic space makes things interesting.

Why are these groups relevant? Let $\Ga \subeq G$ be closed. Let $u\in G_a^-$. Then for all $x=\Ga g\in \Ga\bs G$, $d(a^n.x, a^n.ux)\to 0$ as $n\to \iy$.  To see this, note
\begin{align}
d(a^n.x,a^n.(u.x)) 
&= d(\Ga ga^{-n},\Ga gu^{-1}a^{-n})\\% \xra{n\to \iy} 0\\%mult by inverse on right, always confusing.
&\le d(a^{-n}, u^{-1}a^{-n}) \\
&= d(a^n ua^{-n},e)\xra{n\to \iy}0
\end{align}
%Move forward in time, distance goes to $\iy$
%for some groups ? and this are different
%alg agree - there is a condition to make the P condition algebraic.

What happens over $\SL_2(\R)$? Take $a\in A^+$, $a=\matt{\la}{}{}{\la^{-1}}$. Then
\begin{align}
\matt{\la^n}{}{}{\la^{-n}}\matt abcd \matt{\la^{-n}}{}{}{\la^n} &= \matt a{\la^{2n}b}{\la^{-2c}c}d.
\end{align}
%stable or unstable: $\la>1$ or $<1$
So for $\la>1$, 
\begin{align}
G_a^- &= \bc{\matt 10*1} & P_a^- &= \bc{\matt *0**}\\
G_a^+ &= \bc{\matt 1*01} & P_a^+ &= \bc{\matt **0*}.
\end{align}
%horocycle flow
%geodesic point to same point on boundary

Orbits of $U$ are called horocycles. They are
\begin{enumerate}
\item
horizontal lines
\item
circles tangent to $\R$.
\end{enumerate}•
%dynamics by itsle.f
They are perpendicular to corresponding geodesics.
%sphere is nice picture but you start believing hyperbolic plane is compact, bad idea
%unit tangent bundle... circle bundle...

%famous fundamental domain,
\subsection{Fundamental domain for $\SL_2(\Z)\cir \bb H$}

This is a set $ \cal F\sub \bb H$ such that $\bb H = \bigsqcup_{\ga\in \PSL_2(\Z)} \ga . \cal F$.
One way to prove $\cal F$ is to look at the action under generators---one is inversion, one is translation. It's a classical calculation of 19th century mathematics. It has no way of working for $\SL_{57}(\R)$. 
%siegel domain, is \sqcup shape

%semialgebraic set, important
%semialg fund domain

%existence of semialg important

The fundamental domain is
\begin{align}
\cal F^{\circ} &= \set{z\in \bb H}{-\rc 2<\Re z<\rc 2, |z|>1}\\
\cal F &= \cal F^{\circ} \cup \text{part of boundary}
\end{align}
%thinner because 1/y.
%closed orbit of A, not finite volume
Consider the geodesic that is a vertical line at $i$; it goes to infinity. Most geodesics are dense. Many aren't; their closure can be weird. %If you fold the
But a the closure of a horocyle can only be everything or periodic. If the closure is everything, in the limit it will be equidistributed. 
%some example in mind, draw pic in mind

%be careful
How quickly do closed horocycles equidistribute?  Finding the correct rate is equivalent to the Riemann hypothesis. 
We'll discuss the 
Margulis mixing argument, quantitative nondivergence.
Just by understanding how the Haar measure behaves under $G$, we can understand equidistribution of horocycles; this doesn't work of other groups.

\subsection{Measurable dynamical systems}

The systems we are interested in will always be $(\Ga\bs G, \mu)\cil H$, where $H\le G$ is closed. 

\begin{df}
\begin{enumerate}
\item
An action $G\cir (X,\chi,\mu)$, where $G$ is locally compact and $\chi$ is a probability space ($\si$-algebra), is measure-preserving if %fixed sigma-algebra even if vary the measure
\begin{align}
\forall g\in G, \quad \forall A\in \chi: \quad \mu(g^{-1}.A) = \mu(A)
\end{align}•
%interesting for semigroups.
(A good example: $G$ with Haar measure. The quality of being measure-preserving descends to subgroups.)
\item
A measure-preserving $G\circ (X,\mu)$ is \vocab{ergodic} if any measurable set $A$ satisfying $\mu(g^{-1}.A\triangle A)=0$ %(Any set is essentially invariant.)
for all $g\in G$, then $\mu(A)=0$ or $\mu(X\bs A)=0$. 
\item
A measure-preserving action $G\cir (X,\mu)$ is \vocab{mixing} if for all $A,B$ measurable and any $g_n\to \iy$ in $G$, 
\begin{align}
\mu((g_n^{-1}.A)\cap B) \xra{n\to \iy} \mu(A)\mu(B).
\end{align}
(If you move far away from the action, it becomes independent.)
\end{enumerate}
\end{df}
%countably generated si-algebra
%highly abstract nonsense
%is every si-alg equiv to countable si-alg - axiom of choice...
%people who like explicit integers shouldn't think too much about.
%We can ask for stronger. Equivalent, not trivial to prove.
Mixing implies ergodicity: 
Take $A$ $G$-invariant; then  because applying $g$ many times you get $\mu(A)^2=\mu(A)$. 

Being weakly invariant under a subgroup is a weaker condition. 
Ergodicity does not descend to subgroups; mixing does.
To show ergodic for a subgroup, you can prove mixing for a larger group. 
\subsection{Koopman representation}

Let $G\cir (X,\chi, \mu)$ be a measure-preserving map. There is a unitary representation of $G$, 
\begin{align}
%\pi
U: G &\to U(L^2(X,\mu))\\
%\pi(g) . f
U_g(f) &=f(g^{-1}.x) %= U_g(f),
\end{align}
continuous if $G\times X\to X$ is continuous.
%norm on unitary operators.
It preserves the inner product because it is measure-preserving.
\begin{rem}%read it in the book
\begin{enumerate}
\item
$G\cir X$ is ergodic iff the only eigenfunctions of $G$ in $L^2(X,\mu)$ with eigenvalue 1 is $\C\cdot 1$. %\one %1 is a simple eigenvalue of $G\cir L^2(X,\mu)$. 

This is a fancy way of saying: for all $f\in L^2(X,\mu)$, $G\circ f = f$ implies $f\in \C\times 1$.
\item
$G\cir X$ is mixing if $\forall f,g\in L^2(X,\mu)$,  for all $g_n\to \iy$,
\begin{align}
\an{U_{g_n}f,g} \xra{n\to \iy} \int f\,d\mu\int g\,d\mu.
\end{align}
%asymptotic independence
%mixing implies ergodicity, the other way arround is wrong, irrational rotation on torus is ergodic but not mixing.
\begin{exr}
Let $\bb T^d = \Z^d \bs \R^d$ with the Lebesgue measure. Let $\mu\in \GL_d(\Z)$ be hyperbolic (no eigenvalues of modulus 1). 

$\Z$ acts on $\bb T^d$ by $M^n$, ($v+\Z^d$).  Show that this action is mixing, in fact exponentially mixing, 
\begin{align}
|\an{M^n f,g}-\int f\int g|\ll_{f,g} e^{-\al n}
\end{align}
(Use Fourier analysis.)
\item
Fix $\al\in \R^d$, then $\Z\cir \bb T^d$ by translations. Show this is not mixing.
%interesting actions of lie groups not mixing.
\end{exr}
\end{enumerate}

\end{rem}